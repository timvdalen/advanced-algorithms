\begin{enumerate}[(i)]
	\item
\textsc{Alg} estimates some function $\Phi(\sigma)$ of an input stream $\sigma$, where $\Phi(\sigma) > 3$. It outputs a value $\tilde{\Phi}(\sigma)$ such that $\text{E}[\tilde{\Phi}(\sigma)] = \Phi(\sigma)$ and $\text{Var}[\tilde{\Phi}(\sigma)] = \frac{1}{3} * \Phi(\sigma)$. 
There is a constant $c$ with $0 < c < 1$ such that
\[
\begin{split}
\text{Pr}[|\tilde{\Phi}(\sigma) - \Phi(\sigma)| \geq c * \Phi(\sigma)] &\leq \frac{1}{6}\\
\end{split}
\]
This can be derived from the Chebyshev Inequality as clarified below
\[
\begin{split}
\text{Pr}\left[|X - \mu)| \geq t * \sqrt{\text{var}[X]}\right] \leq \frac{1}{t^2} & \hspace*{20px}\text{Chebyshev Inequality}\\
\text{Pr}\left[|\tilde{\Phi}(\sigma) - \text{E}[\tilde{\Phi}(\sigma)])| \geq t * \sqrt{\text{Var}[\tilde{\Phi}(\sigma)]}\right] \leq \frac{1}{t^2} & \hspace*{20px}\text{$\tilde{\Phi}(\sigma)$ is a random variable}\\
\text{Pr}\left[|\tilde{\Phi}(\sigma) - \Phi(\sigma)| \geq t * \sqrt{\frac{1}{3} * \Phi(\sigma)}\right] \leq \frac{1}{t^2} & \hspace*{20px}\text{}\\
\text{Pr}\left[|\tilde{\Phi}(\sigma) - \Phi(\sigma)| \geq t * \frac{1}{\sqrt{\frac{1}{3} * \Phi(\sigma)}} * \frac{1}{3} * \Phi(\sigma)\right] \leq \frac{1}{t^2} & \hspace*{20px}\text{}\\
\text{Pr}\left[|\tilde{\Phi}(\sigma) - \Phi(\sigma)| \geq c * \Phi(\sigma)\right] \leq \frac{1}{t^2} & \hspace*{20px}\text{with $c = t * \frac{1}{\sqrt{\frac{1}{3} * \Phi(\sigma)}} * \frac{1}{3}$}\\
\end{split}
\]
Since we want the probability to be less than $\frac{1}{6}$ we should set $t$ to $\sqrt{6}$ to determine $c$
\[
\begin{split}
\Phi(\sigma) &> 3\\
\sqrt{\frac{1}{3} * \Phi(\sigma)} &> \sqrt{1} > 0\\
\frac{1}{\sqrt{\frac{1}{3} * \Phi(\sigma)}} &> 0\\
\sqrt{6} * \frac{1}{\sqrt{\frac{1}{3} * \Phi(\sigma)}} * \frac{1}{3} &> 0 \\
\end{split}
\hspace*{20px}
\begin{split}
\Phi(\sigma) &> 3\\
\sqrt{\frac{1}{3} * \Phi(\sigma)} &> \sqrt{1}\\
\frac{1}{\sqrt{\frac{1}{3} * \Phi(\sigma)}} &< 1\\
\sqrt{6} * \frac{1}{\sqrt{\frac{1}{3} * \Phi(\sigma)}} * \frac{1}{3} &< 1 \\
\end{split}
\]
We can conclude that there exists a $c$ for which $0 < c < 1$ holds and for which the probability $\text{Pr}[|\tilde{\Phi}(\sigma) - \Phi(\sigma)| \geq c * \Phi(\sigma)]$ is less or equal to $\frac{1}{6}$.
	\item
By running \textsc{Alg} in $k$ times in parallel and taking the median of the results as $\hat{\Phi}(\sigma)$ we can obtain a probability equal to $1-\delta$.
We define random variables $X_i$ and $Y_i$ for each parallel outcome $i\in \{1, \ldots , k\}$. In addition we define $X$ to be $\sum^k_{i=1}X_i$ and $Y$ to be $\sum^k_{i=1}Y_i$
\[
\begin{split}
X_i &= 
\begin{cases}
1 & \text{if $\tilde{\Phi}_i(\sigma) < (1-c) * \Phi(\sigma)$}\\
0 & \text{otherwise}
\end{cases}\\
Y_i &= 
\begin{cases}
1 & \text{if $\tilde{\Phi}_i(\sigma) > (1+c) * \Phi(\sigma)$}\\
0 & \text{otherwise}
\end{cases}
\end{split}
\]
The expected values of $X$ and $Y$ are $\text{E}[X] = \text{E}[Y] = k*\frac{1}{6} / 2 = k/12$.
If either $X_i  = 1$ or $Y_i = 1$ for more than half of the $k$ reported values then the median will not provide sufficiently accurate output.
The chances $\text{Pr}[|\hat{\Phi}(\sigma) - \Phi(\sigma)| \geq c * \Phi(\sigma)]$ of this happening are calculated as follows.
\[
\begin{split}
\text{Pr}[X > k/2] &= \text{Pr}[X > 6 * k/12]\\
\text{Pr}[X > k/2] &= \text{Pr}[X > 6 * \text{E}[X]]\\
\text{Pr}[X > k/2] &< (\frac{e^5}{6^6})^{k/12} \hspace*{20px}\text{Chernoff bound}\\
\text{Pr}[Y > k/2] &< (\frac{e^5}{6^6})^{k/12} \hspace*{20px}\text{Similarly}\\
\text{Pr}[|\hat{\Phi}(\sigma) - \Phi(\sigma)| \geq c * \Phi(\sigma)] &= 1 - \text{Pr}[X > k/2] - \text{Pr}[Y > k/2]\\
\text{Pr}[|\hat{\Phi}(\sigma) - \Phi(\sigma)| \geq c * \Phi(\sigma)] &> 1 - 2 *  (\frac{e^5}{6^6})^{k/12}\\
\end{split}
\]
To achieve the desired probability we need to set to $k$ such that 
\[
\begin{split}
1 - \delta < 1 - 2 *  (\frac{e^5}{6^6})^{k/12}\\
\end{split}
\]
\end{enumerate}
