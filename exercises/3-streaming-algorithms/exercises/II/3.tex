The following algorithm is able to solve \textsc{Median} for a stream exactly in three passes.
It does so by picking elements in the first pass and ranking these elements in the second pass.
The third pass can use the ranked elements to find bounds for the actual median of which we know the rank, since it's the middle element.
Due to the randomly picked elements the bounded range is likely to be relatively small compared to the entire stream.
The third pass simply stores all elements within this range in a set $O$, then it can calculate the median by using the relative rank in the set and the rank of the bounding elements.

\begin{sourcecode}
\streaming{ThreePassMedian}
\event{Input}
A stream $<a_1, \ldots , a_m>$ from universe $[n]$ in the vanilla model.
\silend

\event{pass 1 - Initialize}
Pick $k$ indices $r_1, \ldots , r_k$ independently and uniformly at random from $\{1, \ldots , m\}$.
Set $R = \{r_1, \ldots , r_k\}$ and $J = \emptyset$ where $R$ and $J$ are considered multisets.
$\textit{min} = n$
$\textit{max} = 0$
\silend

\event{pass 1 - Process$(a_i)$}
\qif{$i \in R$} 
	$J = J \cup \{a_i\}$
|
\qif($a_i < \textit{min}$)
	$\textit{min} = a_i$
|
\qif($a_i > \textit{max}$)
	$\textit{max} = a_i$
|
\silend

\event{pass 2 - Initialize}
$J = J \cup \{min, max\}$
\for{$x \in J$}
	$\textit{rank}(x) = 1$
|
\silend

\event{pass 2 - Process$(a_i)$}
\for{$x \in J$} 
	\qif{$a_i < b$}
		$\textit{rank}(a_i) = \textit{rank}(a_i) + 1$
	|
|
\silend

\event{pass 3 - Initialize}
Determine which of the elements in $J$ can serve as a lower bound $l$ for the median, this element should be closest to but lesser than or equal to $\lfloor (m+1)/2 \rfloor$.
Determine which of the elements in $J$ can serve as a upper bound $u$ for the median, this element should be closest to but greater than or equal to $\lceil (m+1)/2 \rceil$.
$O = \emptyset$
\silend

\event{pass 3 - Process$(a_i)$}
\qif{$l \leq a_i \leq u$}
	$O = O \cup \{a_i\}$
|
\silend

\event{Output}
Report the median of the stream from set $O$ by using the \textit{rank} of $l$ or $u$.
\qend
\end{sourcecode}

\paragraph{Correctness}
If we disregard the storage requirement we can easily determine that the algorithm is correct. Once we have the rank of an element $x$ and a number of subsequent elements we can determine the ranks of these subsequent elements. If we know that $\textit{rank}(x) \leq \lfloor (m+1)/2 \rfloor$ and it holds for one of the subsequent elements $y$ that $\textit{rank}(y) \geq \lceil (m+1)/2 \rceil$ then the median must be among these. Since the algorithm takes it's lower and upper bounds from a set which includes the minimum and the maximum, there are allways bounding elements such that the median is contained within the range.

\paragraph{Storage}
Pass 1 stores two sets with $k$ numbers and two numbers, so $\Theta(k\log_n{n})$ bits.

Pass 2 stores a set with $k+2$ numbers and a rank for $k+2$ items, so again $\Theta(k\log_n{n})$ bits.

In pass 3, the storage size depends on the set $O$.
The size of set $O$ in turn depends on the randomly chosen indices in pass 1.
