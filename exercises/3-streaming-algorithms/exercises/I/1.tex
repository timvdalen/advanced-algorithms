We can use the same line of reasoning as that used for \textbf{Theorem~8.1} in the Course Notes on Streaming Algorithms.

Suppose a deterministic streaming algorithm \textsc{Alg} for \textsc{Element Uniqueness} uses at most $s$ bits of storage.
Then \textsc{Alg} can only be in $2^s$ different states at any point in time.
In particular, \textsc{Alg} can only be in $2^s$ different states after processing the first $m - 1$ tokens in the stream.

On the other hand, the number of different frequency vectors $F[1, \ldots, n]$ that can be generated by a stream of $m - 1$ tokens from $\left[n\right]$ is:

%TODO: Check this
\[
	{n + m - 1 - 1 \choose n - 1} \geq \left(\frac{n + m - 2}{m - 1}\right)^{m - 1} = 2^{(m - 1\log\left(\frac{2\left(n+m - 2\right)}{m}\right))}
\]

Hence, when $s < m - 1\log\left(\frac{2\left(n + m - 2\right)}{m}\right)$ there must be two sequences $\sigma_1 = <a_1, \ldots, a_{m - 1}>$ and $\sigma'_1 = <a'_1, \ldots, a'_{m - 1}>$ with the following property:
$\sigma_1$ and $\sigma'_1$ define different frequency vectors but \textsc{Alg} is in exactly the same state after processing $\sigma_1$ as it would be after processing $\sigma'_1$.

%TODO: Elaborate on construction of the $\sigma$'s

Now let $\sigma_2 = <a_m>$ be a sequence such that $\sigma_1 \circ \sigma_2$ contains a certain $j \in \left[n\right]$ that is a unique element, with $j$ not being a unique element in $\sigma'_1 \circ \sigma_2$.

Since \textsc{Alg} is deterministic and the state of \textsc{Alg} after processing $\sigma_1$ is the same as it would be after processing $\sigma'_1$, we can concluse that \textsc{Alg} will report the same answer for the stream $\sigma_1 \circ \sigma_2$ as it would for $\sigma'_1 \circ \sigma_2$.
But this is incorrect, since $j$ is a unique element for $\sigma_1 \circ \sigma_2$, but not for $\sigma'_1 \circ \sigma_2$.
Hence, any deterministic streaming algorithm that solves \textsc{Element Uniqueness} exactly must use at least $\Theta\left(m \log\left(\frac{2n}{m}\right)\right)$.
