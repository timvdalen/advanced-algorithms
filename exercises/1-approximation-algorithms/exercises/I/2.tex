We know that $m = 10$, $1 \leq t_i \leq 20$ for $1 \leq i \leq n$ and $\sum_{i=1}^n{t_i} \leq 1000$.

This allows us to more clearly define the LB.
From the lemma given in the lecture, we know:

\[
	LB = \max{\left(\frac{1}{m} \sum_{j=1}^n{t_j}, \max_j{t_j} \right)}
\]

Since $\sum_{i=1}^n{t_i} \leq 1000$, we have that $\frac{1}{m}\sum_{i=1}^n{t_i} \leq \frac{1000}{m}$.
Using this and the fact that job sizes are always $\leq 20$, we can see that:

\[
	LB = \max{\left(100, 20\right)} = 100
\]

\paragraph{Theorem} \textsc{GreedyScheduling} for the given configuration is a $1 \frac{9}{50}$-approximation

\paragraph{Proof} Define $m_{i^*}$ as the machine with maximum load. $j^*$ is the last job put on $m_{i^*}$.
$T'(i)$ = load of $m_i$ before job $j^*$ was added.
Makespan = $T(i^*) = T'(i^*) + t_{j^*}$.

From the lecture:
$T'(i^*) \leq \frac{1}{m} \left( \sum_{j=1}^n{t_j} - t_{j^*} \right) \leq LB - \frac{1}{m} t_{j^*} \leq LB - 2 \leq LB - \frac{2}{100} LB$.

$T(i^*) \leq LB - \frac{2}{100} LB + \frac{1}{5} LB \leq 1 \frac{9}{50} LB \leq 1 \frac{9}{50} OPT$.
