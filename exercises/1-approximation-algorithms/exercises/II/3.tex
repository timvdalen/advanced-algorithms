\begin{enumerate}[(i)]
	\item Let $C$ be the set of vertices returned by the algorithm, that is, a set of vertices that form a minimum-size vertex cover.
		Define $c_i$ for each $v_i \in V$ as follows:

		\[
			c_i =
				\begin{cases}
					1   & \text{if } v_i \in C\\
					0   & \text{otherwise}
				\end{cases}
		\]

		We want to minimize the number of vertices in the output set, thus, we want to minimize $\sum_{v_i \in V} c_i$.

		We add the constraint that for every $d$-edge, there should be at least 2 vertices in $C$.

		The linear program follows:

		\linprog{\sum_{v_i \in V} c_i}
			{c_i \in \left\{0, 1\right\}}    {\eqnlimit[1]{i}[n]}
			{\sum_{v \in e}{c_v} \geq 2}    {\text{for all }e \in E}
			{eq:prog:cover}

		Since this is an $0/1$-LP, the constraint (\ref{eq:prog:cover-3}) ensures that there are at least 2 vertices in $C$ per edge.
	\item To give an approximation for the problem, we first convert the program to a normal LP, that is, we change constraint (\ref{eq:prog:cover-2}) to:
		\[
			c_i \in \left\{r \in \mathbb{R}: 0 \leq r \leq 1\right\}, \eqnlimit[1]{i}[n]
		\]

		Let $T$ denote the value of an optimal solution to the relaxed linear problem described above, and let \textsc{OPT} denote the minimum number of vertices in a vertex cover.
		Any vertex cover corresponds to a feasible solution of the linear program, by setting the variables of the vertices in the cover to 1 and the other variables to 0.
		Hence, the optimal solution of the linear program is at least as good as this solution.
		Thus, $\textsc{OPT} \geq T$.

		Then we perform rounding on the resulting LP; variables whose value is at least $\frac{1}{d-2}$ are rounded to 1, variables whose value is less than $\frac{1}{d-2}$ are rounded to 0.
		We thus obtain the following algorithm for \textsc{VertexCover}.

		\begin{sourcecode}
\algorithm{ApproxVertexCover($G=(V,E)$)}
$G$ is a $d$-hypergraph
Solve the relaxed linear program corresponding to the given problem: \linprog{\sum_{v_i \in V} c_i}{c_i \in \left\{r \in \mathbb{R}: 0 \leq r \leq 1\right\}}{\eqnlimit[1]{i}[n]}{\sum_{v \in e}{c_v} \geq 2}{\text{for all }e \in E}{eq:prog:cover-relaxed}
$C = \left\{v_i \in V: c_i \geq \frac{1}{d-2}\right\}$
\return{$C$}
\qend
		\end{sourcecode}

		We first argue that the set $C$ returned by the algorithm is a vertex cover.
		This proof is similar to \textbf{Theorem~2.4} in the course notes.

		Consider an edge $e = (v_1, v_2, \ldots, v_d) \in E$.
		Then, $\sum_{v \in e}{c_v} \geq 2$ is one of the constraints of the linear program.
		A solution will be reported, since the program is obviously feasible by setting all variables to 1.

		Say $\sum_{1}^{\#e - 2}{c_v} < \frac{1}{d-2}$.
		Then, by (\ref{eq:prog:cover-relaxed-3}), $\sum_{v \in e}{c_v} \geq 2$.
		By (\ref{eq:prog:cover-relaxed-3}), $\sum_{\#e - 2}^{\#e}{c_v} \leq 1$.
		Thus, this can not be a solution reported by the relaxed LP.

		The reported solution has at least two vertices $\geq \frac{1}{d-2}$.
		It follows that at least two literals will be put into $C$.

		Let $T$ be $\sum_{v_i \in V} c_i$ be the optimal solution to the relaxed linear program.
		We have $\textsc{OPT} \geq T$.
		Using that $c_i \geq \frac{1}{d-2}$ for all $v_i \in C$, we can now bound the optimal solution as follows:

		\[
			\sum_{v_i \in C} c_i \geq \sum_{v_i \in C} d-2 c_i \geq d-2 \sum_{v_i \in C} c_i \geq \sum_{v_i \in V} c_i = (d-2)T \geq d-2 \textsc{OPT}
		\]

		Thus, \textsc{ApproxVertexCover} is a $d-2$-approxmiation for \textsc{VertexCover}.
	\item
\end{enumerate}
