\begin{enumerate}[(i)]
	\item A suitable value for $\Delta$ can be derived as follows

\begin{sourcecode}
\algorithm{PTAS-TSP}
$\Delta \leftarrow (\varepsilon / n) * UB$ where $UB \leftarrow \sum_{p \in P}{2 * \text{dist}(O, p)}$ \comment{Since Travelling from one point to another will allways add a shorter distance to the path than travelling to the point via the origin $O$.}
\end{sourcecode}

	\item Prove $\textit{length}(T^*) \leq (1 + \varepsilon) * \textit{length}(T_{\textsc{OPT}})$

$\textit{length}(T^*) 

\begin{equation}
\begin{split}
\textit{length}(T^*)  &\leq \\
&\leq \\
&\leq (1 + \varepsilon) * \textit{length}(T_{\textsc{OPT}})
\end{split}
\end{equation}

	\item The algorithm runs in $O(nP^*)$ time.
		Using our $\Delta$, we can obtain a value for $P^*$ in terms of $\varepsilon$ and $n$.

		\begin{align}
			p^*_x &= \ceil{\frac{p_x}{\Delta}}\\
			      &\leq \ldots\\
			      &\leq \ceil{\dots}
		\end{align}

		Similarly, $p^*_y \leq \ceil{\ldots}$.

		Thus, $P^* = O(\ldots)$, and $O(nP^*) = O(n\ldots)$.
		So, the algorithms is a PTAS.
\end{enumerate}
