Consider a permutation algorithm that uses additional external memory blocks while operating on array A.
Since the algorithm is a permutation algorithm, we know that any data that is stored in the external memory, is stored there temporarily.
After the algorithm finishes, all data will be in array A.

Consider a write operation that the algorithm performs on an additional external memory block $\mathcal{B}$.
The algorithm takes the data item $d$ from main memory and puts it in $\mathcal{B}$.
Apparently, the algorithm can not place the item in its final position yet.

We now make the following change to the algorithm.
For each such a write operation, instead of writing to $\mathcal{B}$, we write it to the first available position in $A$.
Due to the definition of $A$ and the fact that we have an item $d$ comining from main memory, there must be an available position.
Now, the algorithm will need to do two more I/O's for this item when it can be processed: it needs to read it in, and write it to its final position.

By making this change, we now satisfy the assumption, since no items are written to an external block outside of $A$ anymore.
We have increased the number of I/Os by a factor 3, but because this factor is a constant, the bound of $\Theta(\textsc{Sort}(n))$ still holds.
