if we sort the sequence of operations stable by value $x_i$ we can obtain a list $L$ where all operations that affect the result of a search operation (insertions and deletions) are grouped with that search operation. We can report whether a number $x_i$ is present in $S$ at the time of the search by running the following algorithm on the sorted list.
\begin{sourcecode}
\algorithm{ReportSearches($L$)}
currentX = null;
\for{$(type_i, x_i) \in L$}
	\qif{$\text{currentX} <> x_i$}
		currentX = $x_i$
		c = 0;
	|
	\qif{$type_i == \textit{Search}$}
		\qif{$c > 0$}
			\textbf{print} $x_i$ is present in $S$ at the time of \textit{Search} $i$
		\qelse
			\textbf{print} $x_i$ is not present in $S$ at the time of \textit{Search} $i$
		|
	\qelse \qif{$type_i == \textit{Insert}$}
		c++;
	\qelse \qif{$type_i == \textit{Delete}$}
		c--; 
	|
|
\qend
\end{sourcecode}

The described algorithm could be easily modified to handle multisets where a number might be inserted multiple times. It takes $O(\textsc{Sort}(n))$ I/Os to sort the list and $O(\ceil{n/B})$ to process the resulting list of operations with \textsc{ReportSearches}. Overall the problem can thus be solved in $O(\textsc{Sort}(n))$ I/Os without using a buffer tree.